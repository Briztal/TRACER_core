\newpage

\section{Introduction}


\subsection{Primary goals and definitions}

Let a machine, whose behaviour we wish to control;\newline

The machine is defined by a set of $n$ actuators, a control system of size $n$, a geometric model, and a kinematic
model;\newline

The control system is coordinate system that is easier to manipulate, or that has more physical meaning than the
actuator's system.\\
For example, a 6DOF arm's actuator system has no direct spatial meaning, and spacial trajectory are very difficult
to express in it.\\
Target positions and speeds will be expressed in this system;\newline

The geometry is the function that will be used to convert control positions to actuators positions;


\subsection{Notations}

Throughout the rest of this document, we will use the following notations :\newline

$E = {i\in \{0, n-1\}}$.\newline

$i$ is the index of an actuator or of an axis of the control system, $E$.\newline

$A_i$ is an actuator.
\begin{itemize}
    \item[-] $u_{i_A}$ is its unit;
    \item[-] $D_{i_A}$, an interval included in $\mathbb{R}$ is its domain;
    \item[-] $p_{i_A}$ is its position expressed in $u_{i_A}$;
    \item[-] $v_{i_A}$ is its speed, expressed in $u_{i_A} \cdot s^{-1}$
    \item[-] $a_{i_A}$ is its acceleration, expressed in $u_{i_A} \cdot s^{-2}$\newline
\end{itemize}

$C_i$ is an axis of the control system.
\begin{itemize}
    \item[-] $u_{i_C}$ is its position;
    \item[-] $D_{i_C}$, an interval included in $\mathbb{R}$ is its domain;
    \item[-] $p_{i_C}$ is its position expressed in $u_{i_C}$;
    \item[-] $v_{i_C}$ is its speed, expressed in $u_{i_C} \cdot s^{-1}$
    \item[-] $a_{i_C}$ is its acceleration, expressed in $u_{i_C} \cdot s^{-2}$\newline
\end{itemize}

$g :  \prod\limits_{i\in E} D_{i_C} \rightarrow \prod\limits_{i\in E} D_{i_A} $,
$ (p_{i_C})_{i \in E}\rightarrow (p_{i_A})_{i \in E}$\newline

is the geometry function, that translates control positions into actuators positions;


\newpage

\subsection{Machine state}

The state of the machine at a given time is defined by the following set of variables

\begin{itemize}
    \item[-] $u_{i_C}$ its control positions;
    \item[-] $v_{i_C}$ its control speeds;
    \item[-] $u_{i_C}$ its actuation positions;
    \item[-] $v_{i_C}$ its actuation speeds;
\end{itemize}

There is a direct relation between control and actuation positions, and control and actuation speeds, given by
the geometry function, but not between actuation positions and speeds, or control positions and speeds;\newline

To change the state of the machine, we execute a movement.\newline

A movement is defined by :
\begin{itemize}
    \item[-] $d_i$ its actuation distances. Distances could also be expressed in the control system,
    but it makes more sense to consider them in the actuation system, as actuators do the final
    movement;
    \item[-] $t$ its duration;
\end{itemize}


\subsection{Physical limitations}

The machine is controlled by actuators, that can be positioned at any point of their domain, without restrictions.
\newline

However, precautions must be taken when attempting to move them. Each type of actuator has its own kind of speed
limitations, often correlated with the geometry;\newline

For a given actuator, part of a machine, variation of the speed must be carefully monitored.
If speed constraints are not met, actuators may halt, or be damaged, which in both case, would compromise the
machine's integrity;\newline
